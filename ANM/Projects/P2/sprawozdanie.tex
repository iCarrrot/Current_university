\documentclass{article}
\usepackage[utf8]{inputenc}
\usepackage{polski}
\usepackage[polish]{babel}
\usepackage{bbm}
\usepackage{graphicx}    
\usepackage{caption}
\usepackage{subcaption}
\usepackage{epstopdf}
\usepackage{amsmath}
\usepackage{amsthm}
\usepackage{hyperref}
\usepackage{url}
\usepackage{comment}
\newtheorem{defi}{Definicja}
\newtheorem{twr}{Twierdzenie}
\usepackage{listings}
\usepackage{float}


\author{Michał Martusewicz 282023}
\date{Wrocław, \today}
\title{\textbf{Interpolowanie krzywych funkcją sklejaną}  \\ Sprawozdanie do zadania P.2.12}

\begin{document}
\maketitle
\section{Wstęp}

Wybrane przeze mnie zadanie polega na zaprogramowaniu interpolacji pewnych krzywych metodą funkcji sklejanej.
 W \S 2 przedstawię ogólnie tą metodę.
 W \S 3 
 W \S 4 
 W \S 5 
 W \S 6 
 W \S 7 podsumuję moje badania.
 

\section{Wybór metody}

\section{}

\section{Porównanie wyników}


\section{Rozbieganie metody}


\section{Pierwiastki wielokrotne}

\section{Wnioski}



\begin{thebibliography}{9}
	\itemsep2pt
	\bibitem{bib1} \url{http://home.agh.edu.pl/~dziembaj/Old/skrypt%20end/podstrony/steffensen.html}
	(ostatni dostęp do strony \today)
	
	\bibitem{bib2} \url{https://en.wikipedia.org/wiki/Steffensen's_method}
	(ostatni dostęp do strony \today)
    \bibitem{bib3} David Kincaid, Ward Cheney. Analiza numeryczna 

   \bibitem{bib5} A. Bjorck G. Dahlquist - Metody numeryczne (przekład S. Paszkowski, wydanie drugie, Warszawa 1987)
    
		
\end{thebibliography}

\end{document}

\section{Wnioski}




\begin{thebibliography}{9}
	\itemsep2pt
	\bibitem{bib1} \url{http://home.agh.edu.pl/~dziembaj/Old/skrypt%20end/podstrony/steffensen.html}
	(ostatni dostęp do strony \today)
	
	\bibitem{bib2} \url{https://en.wikipedia.org/wiki/Steffensen's_method}
	(ostatni dostęp do strony \today)
    \bibitem{bib3} David Kincaid, Ward Cheney. Analiza numeryczna 
    \bibitem{bib3} \url{http://bit.ly/2fgKejw}
	(ostatni dostęp do strony \today)
   \bibitem{bib5} A. Bjorck G. Dahlquist - Metody numeryczne (strona 225)
    
		
\end{thebibliography}

\end{document}